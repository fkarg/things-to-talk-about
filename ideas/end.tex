



%%%%%%%%%%%%%%%%%%%%%%%%%%%%%%%%%%%%%%%%%%%%%%%%%%SECTION%%%%%%%%%%%%%%%%%%%%%%%%%%%%%%%%%%%%%%%%%%%%%%%%%%
\section{Beispiele für Gute Strategie}
%%%%%%%%%%%%%%%%%%%%%%%%%%%%%%%%%%%%%%%%%%%%%%%%%%SECTION%%%%%%%%%%%%%%%%%%%%%%%%%%%%%%%%%%%%%%%%%%%%%%%%%%

\begin{frame}{Reminder: Gute Strategie}
    Gute Strategie besteht aus:
    \begin{itemize}
        \item Einer Leit-Idee
        \item Einem Ziel
        \item Einem Plan an Kohärenten Aktionen, um zu diesem Ziel zu gelangen
    \end{itemize}
\end{frame}

\begin{frame}[c]{Gute Strategie: Beispiel Intel}
    Intel hat einiges Richtig gemacht:
    \begin{itemize}
        \item War ursprünglich Arbeitsspeicher-Hersteller \pause
        \item Radikaler Umschwung z
    \end{itemize}
\end{frame}



\begin{frame}[c]{Gute Strategie: Beispiel SpaceX}
    Wie sah die Raumfahrt-Industrie vor SpaceX aus? \pause


    Es gab:
    \begin{itemize}
        \item die ULA (United Launch Alliance)
        \item Arianespace (ESA-basiert)
        \item Das Russische Kommerzielle Programm
        \item (Das Chinesische und Japanische Raumfahrt-Programm)
    \end{itemize}

\end{frame}



\begin{frame}[c]{Raketen}

    \begin{tabular}{l|l|l|l|l}
        Wer &   Rakete         &   zum LEO &  zum GTO  &  Kosten         \\ \hline
        ULA & Delta IV (Heavy) & 22'560 Kg & 13'400 Kg & 400 Mio \$      \\ \hline
        ULA & Atlas V          & 18'510 Kg &  8'900 Kg & 150 Mio \$      \\ \hline
        ESA & Ariane 5         & 20'000 Kg & 10'500 Kg & 220 Mio \$      \\ \hline
        ESA & Soyuz II         &  8'200 Kg &  3'250 Kg & 60 Mio  \$      \\ \hline
        ILS & Proton-M         & 23'000 Kg &  6'920 Kg & 150 Mio \$      \\
    \end{tabular}


    \footnotesize
    Disclaimer: es ist sehr Schwer Korrekte Preis-Angaben zu finden, da diese meist nicht Öffentlich sind.
    Auch die Traglasten sind nicht exakt Angebbar, da ständig irgendwelche Verbesserungen (zumindest in letzter Zeit)
    gemacht werden.

\end{frame}


\begin{frame}[c]{SpaceX}

    \includegraphics[height=3cm]{comparison.jpg} \\

    \begin{tabular}{l|l|l|l|l}
        Rakete       & zum LEO   & zum GTO   & Traglast zum Mars & Kosten \\ \hline
        Falcon 9     & 22'000 Kg &  8'300 Kg &  4'200 Kg         & 60Mio  \\ \hline
        Falcon Heavy & 54'400 Kg & 22'200 Kg & 13'400 Kg         & 90Mio  \\
    \end{tabular}


    \footnotesize
    Unter der Annahme, dass die Booster / Erste Stufe nicht wiederverwendet wird.
    Andernfalls sind ca. 30\% der Traglast abzuziehen

\end{frame}


%%%%%%%%%%%%%%%%%%%%%%%%%%%%%%%%%%%%%%%%%%%%%%%%%%SECTION%%%%%%%%%%%%%%%%%%%%%%%%%%%%%%%%%%%%%%%%%%%%%%%%%%
\section{Quellen}
%%%%%%%%%%%%%%%%%%%%%%%%%%%%%%%%%%%%%%%%%%%%%%%%%%SECTION%%%%%%%%%%%%%%%%%%%%%%%%%%%%%%%%%%%%%%%%%%%%%%%%%%
\begin{frame}[c,fragile,allowframebreaks]{Quellen}
    Die Folien sind zu finden unter: \\
    \url{https://github.com/blueburningcoder/things-to-talk-about/tree/master/strategy}


    Das Buch, aus dem ich den Vortrag gebastelt hab:

    \begin{thebibliography}{10}
    \beamertemplatebookbibitems
    \bibitem{Richard Rumelt}
        Richard Rumelt
        \newblock {\em Good Strategy / Bad Strategy}.
        \newblock The Difference and Why It Matters \\
                  ISBN: 978-1-78125-154-6
    \beamertemplatearticlebibitems
    \bibitem{Wikipedia}
        Wikipedia
            \newblock {\em Battle of Trafalgar}
            \newblock \url{https://en.wikipedia.org/wiki/Battle\_of\_Trafalgar}
    \bibitem{Wikipedia}
        Wikipedia
            \newblock {\em SpaceX}
            \newblock \url{http://www.spacex.com/}
    \bibitem{Wihipedia}
        Wikipedia
            \newblock {\em Proton-M}
            \newblock \url{https://en.wikipedia.org/wiki/Proton-M}
    \bibitem{Wikipedia}
        Wikipedia
            \newblock {\em Ariane 5}
            \newblock \url{https://en.wikipedia.org/wiki/Ariane\_5}
    \bibitem{Wikipedia}
        Wikipedia
            \newblock {\em Delta IV Heavy}
            \newblock \url{https://en.wikipedia.org/wiki/Delta\_IV}
    \end{thebibliography}
    % required the allowframebreaks for longer lists

\end{frame}


