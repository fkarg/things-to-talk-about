\section{Jetzt}

\subsection{Einführung}

\begin{frame}[c]{Problem: Zu viel zu tun}
    \large
    \begin{aquote}{David Allen}
        When you're no longer drowning, you can think about which way to
        paddle.
    \end{aquote}
    \pnote{Häufig kommt es einem vor, als ob man untergehen würde}
    \pnote{Also: Erstmal müssen wir schwimmen lernen}
\end{frame}


\begin{frame}[c]{Lösung}
    \includegraphics[width=0.6\textwidth]{writing} (Bild von \cite{writing-pic}) \\
    \vspace{0.5cm}
    Schritt Eins: \enquote{Jetzt}-Situation erfassen und strukturieren.
    \pnote{Alles was noch unsere Aufmerksamkeit brauchen wird, aufschreiben}
\end{frame}


\subsection{Unerledigtes}

\begin{frame}[c]{Definition: Unerledigtes}
    \begin{itemize}[<+(1)->]
        \item Antworten, die ausstehen (z.B. Mails)
        \item Angefangene Projekte
        \item Unaufgeräumte Unterhosen
        \item Unbezahlte Steuern (z.B. für Haustiere)
        \item Veranstaltungen, für die man noch etwas vorbereiten muss (z.B. Todos)
        \item Arztbesuche, Reparaturen, Termine
        \item Versprechen, die noch zu erledigen sind
    \end{itemize}
    \pause
    Alles, was noch nicht erledigt ist, oder noch Aufmerksamkeit
    benötigt. Sei konkret, \enquote{Reich werden} zählt nicht.
\end{frame}


\begin{frame}[c]{Hilfsmittel}
    \large
    \begin{itemize}[<+(1)->]
        \item Deine (aktuelle) Todo-Liste (und weitere)
        \item Stift \& Papier (oder anderes Listenerstellungswerkzeug)
        \item GTD Mind Sweep Trigger List \cite{trigger-list} ( \url{https://gettingthingsdone.com/wp-content/uploads/2014/10/Mind_Sweep_Trigger_List.pdf} )
    \end{itemize}
    \pnote{Link schicken}
\end{frame}

\begin{frame}[c]
    \large
    \begin{block}{Aufgabe: Situation erfassen}
    \begin{itemize}
        \item Schreib alles auf, was deine Aufmerksamkeit hat (oder haben sollte!)
        \item Sowohl privat, als auch für's Studium oder die Arbeit!
        \item Erledige alles direkt, was weniger als 2 Minuten braucht!
        \item Ziele kommen auf eine separate Liste!
    \end{itemize}
    \end{block}
    \pnote{Zeit: ~30min}
\end{frame}
\fpause


\subsection{Organisation}

\begin{frame}[c]{Definition: Projekt}
    Ein \blue{Projekt} ist alles, was \textbf{mehr als eine} \green{konkrete
    Aufgabe} benötigt, um abgeschlossen zu werden. \newline \newline \pause
    \textbf{Beispiel:} Wenn man \blue{jemandem ein Geschenk besorgen}
    will, muss man zuerst \green{herausfinden, was die Person mag.}
\end{frame}


\begin{frame}[c]{Definition: Kontext}
    % locations, people
    \orang{Kontext} beschreibt eine \textbf{konkrete Eigenschaft} einer
    \green{Aufgabe} oder einem \blue{Projekt.} Häufig die Abhängigkeit zu einem
    bestimmten \textbf{Ort}, einer \textbf{Umgebung}, oder \textbf{Person}.
    \newline \newline \pause
    \textbf{Beispiele:}
    \begin{itemize}
        \item \orang{Zuhause}, \orang{am PC}
        \item \orang{dauert kurz}, \orang{dauert lang}
        \item \orang{Treffen Mentor}, \orang{Treffen Tobi}
    \end{itemize}
\end{frame}


\begin{frame}[c]{Bemerkungen}
    \large
    Normal ist:
    \begin{itemize}[<+(1)->]
        \item Dir fallen weitere Aufgaben ein, die unerledigt sind ($\rightarrow$ aufschreiben)
        \item Dir fällt auf, dass etwas ein Projekt ist \\ ($\rightarrow$ finde heraus, was die {\em nächste unerledigte Aufgabe} ist)
    \end{itemize}
\end{frame}


\begin{frame}[c]
    \large
    \begin{block}{Aufgabe: Projekte zusammenfassen}
    Fasse thematisch zusammenhängende \green{Aufgaben} zu \blue{Projekten} und
    \orang{Kontexten} zusammen, und markiere zusammengehörende Elemente
    eindeutig (z.B.: Symbol mit einheitlicher Farbe).
    \end{block}
    \pnote{Zeit: 20+ min}
\end{frame}

\fpause


\subsection{Rückblick}

\begin{frame}[c]{Optional: Jahresrückblick}
    \begin{itemize}[<+(1)->]
        \item Zusätzliche, ausführliche Trigger-List
        \item Kann {\em sehr} viel Zeit in Anspruch nehmen
        \item Kann Sinn machen, gleichzeitig Ziele zu sammeln
        \item Ressourcen: \url{https://drive.google.com/file/d/0B2PaeRjVqAN7MngxTXFPQkpLVjg/view} \cite{8760-hours} (Seiten 11-15) (oder anschließend in dieser PDF)
    \end{itemize}
    \pnote{Zeit: 30+ min}
\end{frame}

% \includepdf[pages={11-15},frame,pagecommand={},width=\textwidth]{8760-hours-v2.pdf}
% \includepdf[fitpaper=true,pages=-,frame,pagecommand={},width=\textwidth]{8760-hours-v2.pdf}
{
    \addtocounter{framenumber}{5}
    \setbeamercolor{background canvas}{bg=}
    \includepdf[pages={11-15}]{8760-hours-v2.pdf}
}

\fpause
