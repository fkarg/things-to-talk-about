\section{Weiterer Ablauf}


\begin{frame}[c]{Stand eintragen!}
    \begin{itemize}[<+(1)->]
        \item Falls ihr das noch nicht getan habt:
        \item Tragt euch ein, wenn ihr einen Stand haben wollt!
        \item \textbf{Deadline:} 10. März 2020
        \item Hier: \url{https://docs.google.com/spreadsheets/d/1sObLJ900zgHM6nRhGIgroGbhjrJ0TOvcrC54IP21NJo/edit?usp=sharing}
        \item<handout> Nächstes Treffen wollen wir die Stände verteilen!
    \end{itemize}
\end{frame}


\begin{frame}[c]{Materialien}
    \begin{itemize}[<+(1)->]
        \item Was kommt in euren Cocktail / Was wollt ihr an Deko
        \item Wie viel kostet das zu kaufen/mieten
        \item Wo gibt es das (idealerweise Metro/Baumarkt/...-Link)
        \item Wie viel davon / Stückzahl
    \end{itemize}
\end{frame}


\begin{frame}<handout:0>[c]{Mailverteiler festgruppen@}    
    \begin{itemize}[<+(1)->]
        \item Wichtige Infos ab sofort nur noch über Mailverteiler
        \item Meldet euch bei uns falls ihr noch nicht drauf steht!
    \end{itemize}
\end{frame}


\begin{frame}<handout>[c]{Mailverteiler festgruppen@}
    Weitere wichtige Infos wird es nur über den Festgruppenverteiler geben. \\
    Wenn ihr drauf wollt schreibt uns eine Mail mit eurer Festgruppe an:
    \mailto{festgruppenkoordination@unifest-karlsruhe.de}
\end{frame}


\begin{frame}[c]{Informationskanäle}
    \begin{itemize}[<+(1)->]
        \item Es wird mehrere Informationskanäle geben
        \item Insbesondere mit verschiedenen Personengruppen \textbf{während} dem Fest
        \item Beispielsweise für Getränkenachschub über Funk
        \item Werden alle noch rechtzeitig bekanntgegeben
    \end{itemize}
\end{frame}


\begin{frame}[c]{Optionaler Informationskanal}
    \includegraphics[height=0.8\textheight]{qr-code} \\
    \url{https://t.me/standunifest2020}
\end{frame}




\begin{frame}[c]{Nächstes Treffen}
    \begin{itemize}[<+(1)->]
        \item Am 28. April
        \item Um 18 Uhr
        \item Ort wird noch bekannt gegeben
        \item Wir teilen euch mit, wie wir die Stände verteilt haben
        \item Vermutlich: Einführung in das Helfersystem
        \item<handout> Wichtig für Standbetreuer!
    \end{itemize}
\end{frame}


\begin{frame}[c]{Notwendige Folie}
    ... Damit es 42 sind :)
\end{frame}

