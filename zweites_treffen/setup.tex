
\usepackage{etex}
\usepackage{graphicx}
\usepackage[export]{adjustbox}
\usepackage{multicol}
\usepackage{marvosym}
\usepackage{pdfpcnotes}
\usepackage{pdfpages}
\usepackage{multirow}




% \usepackage{minted}
% \usemintedstyle{pastie}


\usetheme[numbering=fraction, progressbar=frametitle]{metropolis}


% \date{\today}
\date{03. März 2020}


\institute{AStA Unifest Festgruppenkoordination}
% REPLACE with new Unifest Logo
\titlegraphic{\vspace{4cm} \hspace{7cm} \includegraphics[height=3cm]{quadratisch}}

\iftwocols
\AtBeginSection[]
{
    \large
    \begin{frame}{Inhalt}
        \begin{multicols}{2}
            \tableofcontents[currentsection]
        \end{multicols}
        \clearpage
    \end{frame}
}

\AtBeginSubsection[]
{
    \large
    \begin{frame}{Inhalt}
        \begin{multicols}{2}
            \tableofcontents[currentsection,currentsubsection]
        \end{multicols}
        \clearpage
    \end{frame}
}

\else

\AtBeginSection[]
{
    \large
    \begin{frame}{Inhalt}
        \tableofcontents[currentsection]
        \clearpage
    \end{frame}
}

\AtBeginSubsection[]
{
    \large
    \begin{frame}{Inhalt}
        \tableofcontents[currentsection,currentsubsection]
        \clearpage
    \end{frame}
}
\fi


\begin{document}

\maketitle

% multicols from:
% https://tex.stackexchange.com/questions/24343/splitting-toc-into-two-columns-on-single-frame-in-beamer

%%%%%%%%%%%%%%%%%%%%%%%%%%%%%%%%%%%%%%%%%%%%%%%%%%%%%%%%%%%%%%%%%%%%%%%%%%%%%%%%%%%%%%%%%%%%%%%%%%%%%%%%%%%%%%%%%%%

\iftwocols
\begin{frame}{Inhalt}
    \large
    \begin{multicols}{2}
%        \tableofcontents[hidesubsections]
        \tableofcontents[]
    \end{multicols}
    % \clearpage
\end{frame}

\else

\begin{frame}{Inhalt}
    \large
%   \tableofcontents[hidesubsections]
    \tableofcontents[]
    % \clearpage
\end{frame}
\fi


\newcommand{\code}[1]{
    \begin{center}
    \setlength{\fboxrule}{1pt}
    \setlength{\fboxsep}{8pt}
        {\fbox{\parbox{0.81\textwidth}{#1}}}
   \end{center}
}

\newenvironment{codeboxed}[1]
        {\begin{minipage}{\linewidth}\begin{center}#1\\[1ex]\begin{tabular}{|p{\textwidth}|}\hline}
        {\\\hline\end{tabular}\end{center}\end{minipage}}


\newcommand{\backupbegin}{
   \newcounter{finalframe}
   \setcounter{finalframe}{\value{framenumber}}
}

\newcommand{\backupend}{
   \setcounter{framenumber}{\value{finalframe}}
}


\newcommand{\mailto}[1]{
    \href{mailto:#1}{#1}
}


\newcommand{\todo}[1]{
    {\Large\color{red}{(TODO: #1)}}
}

\newcommand{\Q}[1]{
    {\textbf{Q:} #1 \\}
}
\newcommand{\A}[1]{
    {\textbf{A:} #1 \\}
}
