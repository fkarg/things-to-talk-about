
\section{Vorteilsquellen}

\begin{frame}[c]{Die Zukunft Vorhersagen}
    \Huge
    \pause
    Können wir.
\end{frame}


\begin{frame}[c]{Vorteilsarten}
    \Large
    \begin{itemize}
        \item Hebelwirkung
        \item Erreichbare Ziele
            \pause
        \item Starke Position
        \item Hierarchische Ziele
            \pause
        \item Design
        \item Fokus
%        \item (Wachstum)
%        \item (Entropie)
%        \item (Vorteil)
    \end{itemize}
\end{frame}



\begin{frame}[c]{Vorteil: Hebelwirkung}
    \Large
    Kann sein: Situation zum eigenen Vorteil verwenden. \\
    Wichtig: Kritische Punkte (Diagnose!!).
\end{frame}


\begin{frame}[c]{Vorteil: Erreichbare Ziele}
    \Large
    Situation: Mondlandung \\
\end{frame}

\begin{frame}[c]{Vorteil: Starke Position}
    \Large
    Situation: Schachspiel
\end{frame}


\begin{frame}[c]{Vorteil: Hierarchische Ziele}
    \Large
    Situation: Helikopter meistern
\end{frame}


% \begin{frame}[c]{Cached Thoughts: Hierarchisch}
%     \large
%     Leicht unterschiedliche Situation verändert alles
% \end{frame}


\section{Verkettete Systeme}


\begin{frame}[c]{Verkettete Systeme}
    \Large
    Situation: Space Shuttle Challenger \\
    \pause
    Situation: Fähigkeiten
\end{frame}


\begin{frame}[c]{Verkettete Systeme: Verbesserungen}
    \large
    Inkrementelle Verbesserungen helfen (meist) wenig.
\end{frame}


\begin{frame}[c]{Verkettetes System: Beispiel}
    \Large
    Situation: IKEA
    % Einzlne Links:
    %   large inventory
    %   own stores
    %   own designs
    %   catalogue
    %   logistics
    %   ...
\end{frame}




