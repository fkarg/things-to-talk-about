
\begin{frame}[standout]
    Disclaimer: \pause I don't really know what I'm talking about.
\end{frame}

\begin{frame}[c]{Epistemic status}
    \Large
    \begin{itemize}[<+(1)->]
        \item Evolving theory
        \item Hypotheses partially verified
        \item Theories are constantly being updated
        \item This is not the newest information regarding this theory
        \item Bold in its claims
        \item \textbf{Not without criticism}
    \end{itemize}
\end{frame}


\section{What is Intelligence?}


\begin{frame}[c]{Tests for Intelligence}
    \Large
    \begin{itemize}[<+(1)->]
        \item Turing test
        \item 'IQ' tests
        \item Problme solving tests
        \item Car driving skills
        \item ...
    \end{itemize}
    \pause
    But dogs, monkeys and dolphins fail them. \\ \pause
    Focusing on human-like performance is \textbf{limiting.}
\end{frame}




% \begin{frame}[c]{Defining Intelligence}
%     \pause
%     \Huge
%     What does the brain \textbf{do} \newline all the time?
%
%     \vfill
%
%     \phantom{Infer. Predict. Learn.}
% %     Infer. \pause Predict. \pause Learn.
% \end{frame}


\begin{frame}[c]{Intelligence - Definition}
    \Large
    \pause
    Intelligence: The degree of flexibility in both learning and behaviour \cite{hawkins2017book}.
\end{frame}


\begin{frame}[c]{Intelligence - Overview}
    \Large
    \pause
    Might not be best at specific task.

    \pause But can do a lot of different tasks quite well.
    \newline
    \newline
    \pause
    $\rightarrow$ General solution.
\end{frame}


