\documentclass[12pt,compress,english,utf8,t,usenames,dvipsnames]{beamer}
\usepackage[ngerman, english,main=english]{babel}


\title{Hierarchical Temporal Memory}
\subtitle{Biological And Machine Intelligence}
\author{Felix Karg}


\graphicspath{ {../template/} {./graphics/} {../template_tex/} } % add further graphics paths here



\usepackage{etex}
\usepackage{graphicx}
\usepackage[export]{adjustbox}
\usepackage{multicol}


\usepackage{minted}
\usemintedstyle{pastie}


\usetheme[numbering=fraction, progressbar=frametitle]{metropolis}


\date{\today}
% \date{12. Juli 2019}


\institute{Studentische Unternehmensberatung Karlsruhe}
\titlegraphic{\vspace{4cm} \hspace{7cm} \includegraphics[height=1.2cm]{delta-logo}}

\iftwocols
\AtBeginSection[]
{
    \Huge
    \begin{frame}{Content}
        \begin{multicols}{2}
            \tableofcontents[currentsection]
        \end{multicols}
        \clearpage
    \end{frame}
}

\AtBeginSubsection[]
{
    \Huge
    \begin{frame}{Content}
        \begin{multicols}{2}
            \tableofcontents[currentsection,currentsubsection]
        \end{multicols}
        \clearpage
    \end{frame}
}

\else

\AtBeginSection[]
{
    \large
    \begin{frame}{Content}
        \tableofcontents[currentsection]
        \clearpage
    \end{frame}
}

\AtBeginSubsection[]
{
    \large
    \begin{frame}{Content}
        \tableofcontents[currentsection,currentsubsection]
        \clearpage
    \end{frame}
}
\fi


\begin{document}

\maketitle

% multicols from:
% https://tex.stackexchange.com/questions/24343/splitting-toc-into-two-columns-on-single-frame-in-beamer

%%%%%%%%%%%%%%%%%%%%%%%%%%%%%%%%%%%%%%%%%%%%%%%%%%%%%%%%%%%%%%%%%%%%%%%%%%%%%%%%%%%%%%%%%%%%%%%%%%%%%%%%%%%%%%%%%%%

\iftwocols
\begin{frame}{Content}
    \large
    \begin{multicols}{2}
%        \tableofcontents[hidesubsections]
        \tableofcontents[]
    \end{multicols}
    % \clearpage
\end{frame}

\else

\begin{frame}{Content}
    \large
%   \tableofcontents[hidesubsections]
    \tableofcontents[]
    % \clearpage
\end{frame}
\fi


\newcommand{\code}[1]{
    \begin{center}
    \setlength{\fboxrule}{1pt}
    \setlength{\fboxsep}{8pt}
        {\fbox{\parbox{0.81\textwidth}{#1}}}
   \end{center}
}

\newenvironment{codeboxed}[1]
        {\begin{minipage}{\linewidth}\begin{center}#1\\[1ex]\begin{tabular}{|p{\textwidth}|}\hline}
        {\\\hline\end{tabular}\end{center}\end{minipage}}


\newcommand{\backupbegin}{
   \newcounter{finalframe}
   \setcounter{finalframe}{\value{framenumber}}
}

\newcommand{\backupend}{
   \setcounter{framenumber}{\value{finalframe}}
}



% 
\usepackage{etex}
\usepackage{graphicx}
\usepackage[export]{adjustbox}
\usepackage{multicol}


\usepackage{minted}
\usemintedstyle{pastie}


\usetheme[numbering=fraction, progressbar=frametitle]{metropolis}


\date{\today}
% \date{12. Juli 2019}


\institute{Studentische Unternehmensberatung Karlsruhe}
\titlegraphic{\vspace{4cm} \hspace{7cm} \includegraphics[height=1.2cm]{delta-logo}}

\iftwocols
\AtBeginSection[]
{
    \Huge
    \begin{frame}{Content}
        \begin{multicols}{2}
            \tableofcontents[currentsection]
        \end{multicols}
        \clearpage
    \end{frame}
}

\AtBeginSubsection[]
{
    \Huge
    \begin{frame}{Content}
        \begin{multicols}{2}
            \tableofcontents[currentsection,currentsubsection]
        \end{multicols}
        \clearpage
    \end{frame}
}

\else

\AtBeginSection[]
{
    \large
    \begin{frame}{Content}
        \tableofcontents[currentsection]
        \clearpage
    \end{frame}
}

\AtBeginSubsection[]
{
    \large
    \begin{frame}{Content}
        \tableofcontents[currentsection,currentsubsection]
        \clearpage
    \end{frame}
}
\fi


\begin{document}

\maketitle

% multicols from:
% https://tex.stackexchange.com/questions/24343/splitting-toc-into-two-columns-on-single-frame-in-beamer

%%%%%%%%%%%%%%%%%%%%%%%%%%%%%%%%%%%%%%%%%%%%%%%%%%%%%%%%%%%%%%%%%%%%%%%%%%%%%%%%%%%%%%%%%%%%%%%%%%%%%%%%%%%%%%%%%%%

\iftwocols
\begin{frame}{Content}
    \large
    \begin{multicols}{2}
%        \tableofcontents[hidesubsections]
        \tableofcontents[]
    \end{multicols}
    % \clearpage
\end{frame}

\else

\begin{frame}{Content}
    \large
%   \tableofcontents[hidesubsections]
    \tableofcontents[]
    % \clearpage
\end{frame}
\fi


\newcommand{\code}[1]{
    \begin{center}
    \setlength{\fboxrule}{1pt}
    \setlength{\fboxsep}{8pt}
        {\fbox{\parbox{0.81\textwidth}{#1}}}
   \end{center}
}

\newenvironment{codeboxed}[1]
        {\begin{minipage}{\linewidth}\begin{center}#1\\[1ex]\begin{tabular}{|p{\textwidth}|}\hline}
        {\\\hline\end{tabular}\end{center}\end{minipage}}


\newcommand{\backupbegin}{
   \newcounter{finalframe}
   \setcounter{finalframe}{\value{framenumber}}
}

\newcommand{\backupend}{
   \setcounter{framenumber}{\value{finalframe}}
}






\newif\ifonline
\onlinefalse
% \onlinefalse


%%%%%%%%%%%%%%%%%%%%%%%%%%%%%%%%%%%%%%%%%%%%%%%%%%BEGINNING%%%%%%%%%%%%%%%%%%%%%%%%%%%%%%%%%%%%%%%%
% % \section{Introduction}

\begin{frame}[c]{Felix Karg}

    \begin{multicols}{2}
    \begin{itemize}[<+(1)->]
        \item Computer Science (M.Sc.)
    \end{itemize}
    \end{multicols}
\end{frame}



\begin{frame}[c]{Goals for this Talk}
    You know ...
    \begin{itemize}[<+(1)->]
        \item what aging is
        \item why it is a problem
        \item why it is not necessary
        \item how it can be slowed down
        \item about personal anti-aging strategies
        \item how bioinformatics is helping research
    \end{itemize}
\end{frame}


\begin{frame}[standout]
    Disclaimer: \pause I don't really know what I'm talking about.
\end{frame}

\begin{frame}[c]{Epistemic status}
    \Large
    \begin{itemize}[<+(1)->]
        \item Evolving theories
        \item Hypotheses partially verified
        \item Theories are constantly being updated
        \item I know too little to say this is the newest information
    \end{itemize}
\end{frame}


\section{What is Intelligence?}


\begin{frame}[c]{Tests for Intelligence}
    \Large
    \begin{itemize}[<+(1)->]
        \item Turing test
        \item 'IQ' tests
        \item Problme solving tests
        \item Tests for behaviour
        \item ...
    \end{itemize}
\end{frame}


\begin{frame}[c]{Defining Intelligence}
    \pause
    \Huge
    What does the brain \textbf{do} \newline all the time?

    \vfill
    \pause

    \phantom{Infer. Predict. Learn.}
%     Infer. \pause Predict. \pause Learn.
\end{frame}


\input{chapters/bio}
\section{Overview}
\begin{frame}[c]{Plan}
    Individual Parts:
    \begin{itemize}[<+(1)->]
        \item Normal FeedForward MLP
        \item Embedding: Input
        \item Embedding: Location
        \item Basics of Attention (before transformer)
        \item Attention is All You Need \cite{vaswani_attention_2017}
        \item Linear Attention: FastFormer \cite{wu_fastformer_2021} / Flash Attention \cite{hua_transformer_2022}

        \item (fun:) One Model to Learn Them All \cite{kaiser_one_2017}
        \item Distillation / Quantization \cite{polino_model_2018}
    \end{itemize}
\end{frame}



\section{Core Concepts}

\subsection{Hierarchy}


\begin{frame}[c]{Why Hierarchy?}
    \Large
    \pause
    If there is a connection cost, hierarchies are more efficient \cite{mengistu2016evolutionary}. 
    
    \pause
    Especially when tasks change regularly.
\end{frame}


\begin{frame}[c]{Why Hierarchy? II}
    \Large
    \begin{itemize}[<+(1)->]
        \item Reduced Training Time
        \item Reduced Memory Usage
        \item Introduce Generalizations
        \item Learned patterns are recombined at higher levels
        \item Transfer Learning
    \end{itemize}
\end{frame}


\begin{frame}[c]{What Hierarchy}
    \pause
                                        % trim = left bottom right top
    \includegraphics[width=\textwidth, trim = 0 55 0 60, clip]{hierarchy}
\end{frame}


\begin{frame}[c]{Example Application}
    \includegraphics[height=0.9\textheight]{hierarchy_2} 
\end{frame}


\begin{frame}[c]{How Many Levels?}
    \Large
    \begin{itemize}[<+(1)->]
        \item They always learn the best representation
        \item Tradeoff between depth and layer size
        \item Simple problems can be solved with one region
    \end{itemize}
\end{frame}



\subsection{Regions}


\begin{frame}[c]{Region - Introduction}
    \pause
                                        % trim = left bottom right top
    \includegraphics[width=\textwidth, trim = 0 55 0 54, clip]{region}
\end{frame}


\begin{frame}[c]{Region - Details}
    \includegraphics[width=\textwidth]{region_2}
\end{frame}


\begin{frame}[c]{Region - Attributes}
    \Large
    \begin{itemize}[<+(1)->]
        \item All Regions do basically the same
        \item Based on Biological Regions in the Brain
        \item HTM Regions are similar to Layer 3 of the Neocortex
        \item Can do Inference and Prediction even on complex data
    \end{itemize}
\end{frame}




\subsection{Sparse Distributed Representation}
% \subsection{The Datastructure of the Brain}


\begin{frame}[c,fragile]{Data Saving - Computer Science Solution}
    \Large
    What is \verb|01100101|? \pause Could be either one of:
    % What is ? Could be:
    \begin{itemize}[<+(1)->]
        \item Booleans (\verb|False, True, True, False,|\dots)
        \item Integer (\verb|101|)
        \item Float (\verb|3328|)
        \item (Byte-) String (\verb|'e'|)
        \item Pointer to something else
        \item Part of some other Datastructure
    \end{itemize}
\end{frame}

% \begin{frame}[c]{Primitive Computer Science Data Formats}
%     \Large
%     \begin{itemize}[<+(1)->]
%         \item Boolean
%         \item Integer
%         \item Float
%         \item (Byte-) String
%     \end{itemize}
% 
%     \vspace{0.5cm}
% 
%     \pause
% 
%     $\rightarrow$ Data is clearly seperated from Encoding, Meaning comes from actual Encoding
% \end{frame}


\begin{frame}[standout]
    \Large
    Biologically, this does not work out.

    \pause
    We use only 10\% of our Brain, right?
\end{frame}


\begin{frame}[c]{Sparse Distributed Representation - Example}
    \pause
    \includegraphics[width=0.95\textwidth]{region_sparse}
    % \includegraphics[width=0.9\textwidth]{region_predict}
\end{frame}


\begin{frame}[c]{Sparse Distributed Representation - Introduction}
    \Large
    \begin{itemize}[<+(1)->]
        \item Datastructure of the Brain
        \item Sparse (around 2\% are active)
        \item Distributed (clusters are somewhat rare)
        \item Inhibitory Mechanisms
        \item Neuron States actually have 'Meaning'
    \end{itemize}
\end{frame}





% \begin{frame}[c]{Bit Arrays}
%     \Large
%     \pause
%     % Who can code ... ?
% \end{frame}


% \begin{frame}[c]{Attributes of the Brain}
%     \Large
%     \begin{itemize}[<+(1)->]
%         \item Invariant
%         \item Auto-associative
%         \item Massively Parallel
%     \end{itemize}
% \end{frame}





\section{Learning}

\subsection{Overview}

\begin{frame}[c]{Learning}
    % \large
    \begin{itemize}[<+(1)->]
        \item Learning is purely statistical
        \item Lookign for Spatial and Temporal Patterns
        \item Regions themselves are limited
        \item Automatically adjusts to size of allocated Memory
        \item Automatic On-Line learning
        \item takes longer to learn high-level concepts with lower levels missing
        \item only a precursor for inference and prediction
    \end{itemize}
\end{frame}


\begin{frame}[c]{Inference}
    \Large
    \begin{itemize}[<+(1)->]
        \item Matching previously learned sequences
        \item Example: recognizing a Melody
        \item There are only novel experiences
        \item Partial SDR matches
    \end{itemize}
\end{frame}


\begin{frame}[c]{Prediction}
    \Large
    \begin{itemize}[<+(1)->]
        \item Matching stored sequences
        \item Can be thought of to be similar to a markov chain
        \item Takes up a considerable amount of memory
        \item Integral to how the brain works
    \end{itemize}
\end{frame}


\begin{frame}[c]{Prediction - Key Properties}
    \begin{itemize}[<+(1)->]
        \item Continuity
        \item Occurrs everywhere
    \end{itemize}
\end{frame}





\subsection{Spatial Pooling}


\begin{frame}[c]{Spatial Pooler - Introduction}
    \pause
                                     % trim = left bottom right top
    \includegraphics[width=\textwidth, trim= 0 0 97 70, clip]{spatial_pooler} \\
    \normalsize
    Image adapted from \cite{cui2017htm}.
\end{frame}


\begin{frame}[c,fragile]{Spatial Pooler - Connection details}
    \Large
    Show \verb!Ep8/Learning Rules!!
    \newline
    \begin{itemize}[<+(1)->]
        \item Many Connections
        \item Only Columns with highest overlap scores continue
        \item Everyone else gets inhibited
        \item Next: Updating Permanence Values
    \end{itemize}
\end{frame}


\begin{frame}[c,allowframebreaks]{Spatial Pooler - Learning Details}
    \includegraphics[width=\textwidth]{learn_ex5}
    \includegraphics[width=\textwidth]{learn_ex3}
    \includegraphics[width=\textwidth]{learn_ex2}
    \includegraphics[width=\textwidth]{learn_ex1}
\end{frame}



% Show Pictures about granularity with vs without boosting, explain that with boosting more cells start to represent slightly different concepts


\begin{frame}[c,allowframebreaks]{Spatial Pooler - Boosting}
    \includegraphics[width=\textwidth]{boost_ex1}
    \includegraphics[width=\textwidth]{boost_ex2}
    \includegraphics[width=\textwidth]{boost_ex3}
\end{frame}


\begin{frame}[c]{Spatial Pooler - Parameters}
    \Large
    \begin{itemize}[<+(1)->]
        \item Algorithm Structure (receptive field)
        \item Inhibition
        \item Learning rates
        \item Column Activity
    \end{itemize}
\end{frame}


\begin{frame}[c]{Spatial Pooler - Phases}
    \Large
    \begin{enumerate}[<+(1)->]
        \item Initializing with random variables
        \item Compute overlap scores (+Boost)
        \item Inhibition
        \item Updating Permanence values
        \item Repeat from step 2 with new input
    \end{enumerate}
\end{frame}



% SPATIAL POOLER STEPS
% 
% 1. Start with an input consisting of a fixed number of bits. These bits might
% represent sensory data or they might come from another region elsewhere in the
% HTM system.
% 
% 2. Initialize the HTM region by assigning a fixed number of columns to the
% region receiving this input. Each column has an associated dendritic segment,
% serving as the connection to the input space. Each dendrite segment has a set
% of potential synapses representing a (random) subset of the input bits. Each
% potential synapse has a permanence value. These values are randomly initialized
% around the permanence threshold. Based on their permanence values, some of the
% potential synapses will already be connected; the permanences are greater than
% than the threshold value.
% 
% 3. For any given input, determine how many connected synapses on each column
% are connected to active (ON) input bits. These are active synapses.
% 
% 4. The number of active synapses is multiplied by a “boosting” factor, which is
% dynamically determined by how often a column is active relative to its
% neighbors.
% 
% 5. A small percentage of columns within the inhibition radius with the highest
% activations (after boosting) become active, and disable the other columns
% within the radius. The inhibition radius is itself dynamically determined by
% the spread of input bits. There is now a sparse set of active columns.
% 
% 6. The region now follows the Spatial Pooling (Hebbian-style) learning rule:
% For each of the active columns, we adjust the permanence values of all the
% potential synapses. The permanence values of synapses aligned with active input
% bits are increased. The permanence values of synapses aligned with inactive
% input bits are decreased. The changes made to permanence values may change some
% synapses from being connected to unconnected, and vice-versa.
% 
% 7. For subsequent inputs, we repeat from step 3.


% \cite{cui2017htm} % <- spatial pooler




\subsection{Temporal Pooler}


\begin{frame}[c]{Temporal Pooler - Pipeline}
    \includegraphics[width=\textwidth, trim = 0 160 0 0,clip]{spatial_pooler} \\
    \normalsize
    Image adapted from \cite{cui2017htm}.
\end{frame}


\begin{frame}[c]{Temporal Pooler - Introduction}
    \includegraphics[height=0.9\textheight]{sdr_example}
\end{frame}


\begin{frame}[c]{Temporal Pooler - Introduction}
    \includegraphics[width=0.95\textwidth]{region_sparse}
\end{frame}


\begin{frame}[c]{Temporal Pooler - Steps}
    \Large
    \begin{enumerate}[<+(1)->]
        \item Form representation in context of previous states
        \item Form predictions based on previous inputs
    \end{enumerate}
\end{frame}


\begin{frame}[c]{Temporal Pooler - Selecting Winner Cells}
    \includegraphics[height=0.9\textheight]{one_left}
\end{frame}

\begin{frame}[c]{Temporal Pooler - Selecting Winner Cells}
    \includegraphics[height=0.9\textheight]{one_left_burst}
\end{frame}

\begin{frame}[c]{Temporal Pooler - Selecting Winner Cells}
    \includegraphics[height=0.9\textheight]{one_left_winner}
\end{frame}

\begin{frame}[c]{Temporal Pooler - Selecting Winner Cells}
    \includegraphics[height=0.9\textheight]{two_left_middle}
\end{frame}

\begin{frame}[c]{Temporal Pooler - Selecting Winner Cells}
    \includegraphics[height=0.9\textheight]{two_left_middle_burst}
\end{frame}

\begin{frame}[c]{Temporal Pooler - Selecting Winner Cells}
    \includegraphics[height=0.9\textheight]{two_left_last_middle_burst}
\end{frame}

\begin{frame}[c]{Temporal Pooler - Selecting Winner Cells}
    \includegraphics[height=0.9\textheight]{two_left_last_middle_burst_arrow}
\end{frame}

\begin{frame}[c]{Temporal Pooler - Selecting Winner Cells}
    \includegraphics[height=0.9\textheight]{two_left_last_middle_one_arrow}
\end{frame}

\begin{frame}[c]{Temporal Pooler - Selecting Winner Cells}
    \includegraphics[height=0.9\textheight]{three}
\end{frame}

\begin{frame}[c]{Temporal Pooler - Selecting Winner Cells}
    \includegraphics[height=0.9\textheight]{three_burst}
\end{frame}

\begin{frame}[c]{Temporal Pooler - Selecting Winner Cells}
    \includegraphics[height=0.9\textheight]{three_last_burst}
\end{frame}

\begin{frame}[c]{Temporal Pooler - Selecting Winner Cells}
    \includegraphics[height=0.9\textheight]{three_last_active}
\end{frame}

\begin{frame}[c]{Temporal Pooler - Selecting Winner Cells}
    \includegraphics[height=0.9\textheight]{three_last_active_arrow}
\end{frame}

\begin{frame}[c]{Temporal Pooler - Selecting Winner Cells}
    \includegraphics[height=0.9\textheight]{three_without_active}
\end{frame}

\begin{frame}[c]{Temporal Pooler - Selecting Winner Cells}
    \includegraphics[height=0.9\textheight]{three_predicting}
\end{frame}

\begin{frame}[c]{Temporal Pooler - Selecting Winner Cells}
    \includegraphics[height=0.9\textheight]{three_predicting_2}
\end{frame}




% \begin{frame}[c]{Temporal Memory - Examples}
%     \Large
%     \includegraphics[width=\textwidth]{multiple_cols}
%     \vfill
% \end{frame}


\begin{frame}[c]{Temporal Memory - Example I}
    \Large

    \begin{tabular}{llll}
        \includegraphics[width=0.23\textwidth]{active_we} &
        \includegraphics[width=0.23\textwidth]{active_are_2} &
        \includegraphics[width=0.23\textwidth]{active_very_2} &
        \includegraphics[width=0.23\textwidth]{active_busy} \\
        We & are & very & busy
    \end{tabular}

    \pause
    \vfill

    \begin{tabular}{llll}
        \includegraphics[width=0.23\textwidth]{active_you} &
        \includegraphics[width=0.23\textwidth]{active_are_1} &
        \includegraphics[width=0.23\textwidth]{active_very_1} &
        \includegraphics[width=0.23\textwidth]{active_knowledgeable} \\
        You & are & very & knowledgeable
    \end{tabular}
\end{frame}


\begin{frame}[c]{Temporal Memory - Example I}
    \Large
    \includegraphics[width=\textwidth]{active_we_predicted_are2} \\
    We
\end{frame}


\begin{frame}[c]{Temporal Memory - Example I}
    \Large
    \includegraphics[width=\textwidth]{active_are2_predicted_very2} \\
    We are
\end{frame}


\begin{frame}[c]{Temporal Memory - Example I}
    \Large
    \includegraphics[width=\textwidth]{active_very2_predicted_busy} \\
    We are very
\end{frame}


\begin{frame}[c]{Temporal Memory - Example I}
    \Large
    \includegraphics[width=\textwidth]{active_busy} \\
    We are very busy
\end{frame}


\begin{frame}[c]{Temporal Memory - Example I}
    \Large

    \begin{tabular}{llll}
        \includegraphics[width=0.23\textwidth]{active_we} &
        \includegraphics[width=0.23\textwidth]{active_are_2} &
        \includegraphics[width=0.23\textwidth]{active_very_2} &
        \includegraphics[width=0.23\textwidth]{active_busy} \\
        We & are & very & busy
    \end{tabular}

    \begin{tabular}{llll}
        \includegraphics[width=0.23\textwidth]{active_you} &
        \includegraphics[width=0.23\textwidth]{active_are_1} &
        \includegraphics[width=0.23\textwidth]{active_very_1} &
        \includegraphics[width=0.23\textwidth]{active_knowledgeable} \\
        You & are & very & knowledgeable
    \end{tabular}
    \vfill
    \only<1>{\phantom{\includegraphics[width=\textwidth]{active_are_bursting} \\ are}}
    \only<2>{\includegraphics[width=\textwidth]{active_are_bursting} \\ are}
    \only<3>{\includegraphics[width=\textwidth]{active_are_bursting_predict} \\
    are}
    \only<4>{\includegraphics[width=\textwidth]{active_very_predict_two} \\
    very}
\end{frame}


\begin{frame}[c]{Temporal Memory - Advanced}
    \Large
    \begin{itemize}[<+(1)->]
        \item Within a region
        \item But also across
        \item Up and Down
        \item There are much more connections DOWN than UP
        \item In fact, about 90\% go either sideways or DOWN
    \end{itemize}
\end{frame}


\begin{frame}[c]{Temporal Memory - Example II}
    \Large
    \begin{itemize}[<+(1)->]
        \item I \textbf{ate} a pear
        \item I have \textbf{eight} pears
    \end{itemize}
    \vfill
    \begin{itemize}[<+(1)->]
        \item I ...
        \item I have ...
    \end{itemize}
    \pause
    Temporal predictions add to the threshold for the spatial pooler!
    % example for predicting different concepts, included in spatial pooler counting as well
\end{frame}


\begin{frame}[c]{Temporal Memory - SDRs Advanced II}
    \Large
    \textbf{Q:} If you have an SDR with 10 000 Cells and 200 active, how much difference would saving only 20 of them make?
    \newline
    \pause
    \newline
    \textbf{A:} Due to the property of SDRs, it is {\em very} unlikely that they activate in a totally unrelated pattern.
\end{frame}







%         \item If there are no exact matches (would have been predicted and direct winner otherwise), take near matches
%         \item No matching segments at all: select cell with fewest segments (randomly select in case of tie)
%                 Create or grow synapses/segments to winning cells from last step













% \begin{frame}[c]{Temporal Pooler - Detailed Steps}
%     \Large
%     \begin{enumerate}[<+(1)->]
%         \item Compare active columns with prediction. Choose active cells.
%     \end{enumerate}
% \end{frame}


% pictures ...
% \begin{frame}[c]{Temporal Pooler - Collect}
%     \includegraphics[height=0.9\textheight]{temporal_three_none}
% \end{frame}
%
%
% \begin{frame}[c]{Temporal Pooler - Setup}
%     \includegraphics[height=0.9\textheight]{proximal_dendrite_ex1}
% \end{frame}
%
%
% \begin{frame}[c]{Temporal Pooler - Collect}
%     \includegraphics[height=0.9\textheight]{temporal_one_left}
% \end{frame}
%
%
% \begin{frame}[c]{Temporal Pooler - Setup}
%     \includegraphics[height=0.9\textheight]{proximal_dendrite_ex2}
% \end{frame}
%
%
% \begin{frame}[c]{Temporal Pooler - First Bursting Activation}
%     \includegraphics[height=0.9\textheight]{winner_one_all}
% \end{frame}
%
%
% \begin{frame}[c]{Temporal Pooler - Deciding on Winner Cell}
%     \includegraphics[height=0.9\textheight]{winner_one_one}
% \end{frame}




% basically more for documentation:
% \begin{frame}[c]{Temporal Pooler - Rules}
%     \begin{itemize}[<+(1)->]
%         \item If there are no exact matches (would have been predicted and direct winner otherwise), take near matches
%         \item No matching segments at all: select cell with fewest segments (randomly select in case of tie)
%                 Create or grow synapses/segments to winning cells from last step
%     \end{itemize}
% \end{frame}
%
%
% \begin{frame}[c]{Temporal Pooler - Closing notes}
%     \begin{itemize}[<+(1)->]
%         \item Every Column will have a winner cell
%         \item Bursting can happen on single columns as well
%     \end{itemize}
% \end{frame}




\section{Implications}


% - Explain biases based on these models
\begin{frame}[c]{Implications - Models}
    \Large
    Your brain is creating predictive models for everything. \\ \\
    \pause
    You are only familiar with what you have already thought about. \\ \\
    \pause
    Learning is the active creation of new models (spatial and temporal patterns).
\end{frame}


% - requiring ten times more resources during learning
\begin{frame}[c]{Implications - Learning}
    \Large
    While Learning, you need about 90 percent more capacity than after having mastered something. \\ \\
    \pause
    % - AHA-Moment (aligning models, ceasing bursting)
    'AHA-Moment' is when models line up, and bursting ceases. \\ \\
    \pause
    % - Explain the fact that smarter people have less Brain activity
    Smarter people have clearer models and have less overall brain activity (higher efficiacy).
\end{frame}


% - explain gradient for Einstein/Craziness, proving vs disproving in the brain -> THC is bad
\begin{frame}[c]{Implications - Brain Parameters}
    \Large
    There is also a varying threshold for the spatial pooler activations. \\ \\
    \pause
    This probably also changes during sleep.
\end{frame}


% - explain 'top-down' phenomena and common biases

\begin{frame}[c]{Implications - Conformation Bias}
    \Large
    Conformation Bias is a predictable result from the temporal pooler.
\end{frame}


% - Mistakes that happen by top-down enforcing (e.g. the dalmatine picture, scrambled words)
\begin{frame}[c]{Implications - Inhibition in the Visual System}
    \includegraphics[width=\textwidth]{rabbit_duck}
\end{frame}


% - Context-dependent processing
\begin{frame}[c]{Implications - Inhibition in the Visual System}
    \includegraphics[height=\textheight]{The_Dress}
\end{frame}



\begin{frame}[c]{Implications - Top-Down Processing I}
    \Large
    Yuo cna porbalby raed tihs esaliy desptie teh msispeillgns. \\ \\
    \pause
    A vheclie epxledod at a plocie cehckipont near the UN haduqertares in Bagahdd on Mnoday kilinlg the bmober and an Irqai polcie offceir
\end{frame}


\begin{frame}[c]{Implications - Top-Down Processing II}
    \Large
    Aoccdrnig to a rscheearch at Cmabrigde Uinervtisy, it deosn’t mttaer in waht oredr the ltteers in a wrod are, the olny iprmoetnt tihng is taht the frist and lsat ltteer be at the rghit pclae. The rset can be a toatl mses and you can sitll raed it wouthit porbelm. Tihs is bcuseae the huamn mnid deos not raed ervey lteter by istlef, but the wrod as a wlohe.
\end{frame}


% - explain gradient for Autism and that it is more 'bottom up'
\begin{frame}[c]{Implications - Bottom-Up-Gradient}
    \Large
    There is a parameter for how much predictive states count for the spatial pooler. \\ \\
    \pause
    Which one would be better? \\ \\
    \pause
    Possibility: Predictive states count less for Autistic people.
\end{frame}


% - explain theory: inhibited-boost: hallucination
\begin{frame}[c]{Implications - Hallucinating}
    \Large
    Hallucinating is just active boosting in the human brain. \\ \\
    \pause
    Concentration going awry as well ... \\ \\
    \pause
% - Phantom limb 'pain' -> boosting
    Phantom limbs ...
\end{frame}


\begin{frame}[c]{Implications - Reticular Activating System}
    \Large
    Ever noticed after buying something, \\
    everyone has the same thing?
\end{frame}


% - Explain Priming based on predictive stuff (the priming-bias)
\begin{frame}[c]{Implications - Priming Bias}
    \Large
    \pause
    Priming really leaves some predictive states, even though it never gets
    high enough in the hierarchy to notice it consciously.
\end{frame}


% % - Explain the bayesian-reasioning part
% \begin{frame}[c]{Implications - Bayesian Reasoning}
%     \Large
%     The brain is constantly weighing alternatives and selecting the likeliest - as far as it is aware.
% \end{frame}


\begin{frame}[c]{Implications - Cached Thought}
    This fully explains the fallacy of cached thought, among others.
\end{frame}


% - Implications on Meaning (?)
\begin{frame}[c]{Implications - Meaning}
    \Large
    Something has meaning if it is a well-connected concept.
\end{frame}





% - Mistakes that happen by accidential correlation: search for good example; mistaking something to be true



% - Introduce HTMv3 !
% - Motor movements are just very strong predictions the body 'makes happen'
%   - Flow state is where everything is exactly as predicted
% - Location-based composition: This explains mind palaces!
% - Everything is a 'conceptspace' -
% - Explain how cortical columns create quite powerful models



% 
% % \section{HTM Recap}

\begin{frame}[c]{Biology}
    \Large
    \begin{itemize}[<+(1)->]
        \item The Neocortex is only a part of the brain
        \item The smallest computational units are called 'Cortical Columns'
        \item Every Neuron has on average 7'000 Connections 
    \end{itemize}
\end{frame}


\begin{frame}[c]{HTM Theory}
    \Large
    \begin{itemize}[<+(1)->]
        \item Biologically constrained, based on Neuroscience
        \item Everything is predicting everything
        \item Closest to how the brain really works
        \item Time is important
    \end{itemize}
\end{frame}


\begin{frame}[c]{Sparse Distributed Representation}
    \Large
    \begin{itemize}[<+(1)->]
        \item Data structure of the Brain
        \item Sparse (very few ON-bits)
        \item Distributed
        \item Required for other mechanisms
        \item Encoders are important!
    \end{itemize}
\end{frame}


\begin{frame}[c]{Learning}
    \Large
    \begin{itemize}[<+(1)->]
        \item Only statistical
        \item Based on Prediction and Inference
        \item Spatial and Temporal patterns
        \item Context sensitive
        \item Trying to minimize Errors (Bursting)
    \end{itemize}
\end{frame}


\begin{frame}[c]{Spatial Pooler}
    \includegraphics[width=\textwidth]{learn_ex2}
\end{frame}

\begin{frame}[c]{Spatial Pooler}
    \Large
    \begin{itemize}[<+(1)->]
        \item Calculate Overlap Scores (+Boosting)
        \item Inhibit
        \item Update connection values
    \end{itemize}
\end{frame}


\begin{frame}[c]{Temporal Pooler}
    \Large
    \begin{itemize}[<+(1)->]
        \item Form representations in context of previous states
        \item Form predictions based on previous inputs
        \item Context dependent
        \item Can modulate spatial pooler results
        \item One cell can represent many different concepts!
    \end{itemize}
\end{frame}






% 
\section{Open Questions}


\begin{frame}[c]{Open Questions}
    
\end{frame}






% 
\begin{frame}[c]{Relation to Machine Learning}
    \Large
    \pause
    Who has knowledge about Machine Learning? \newline \newline
    \pause
    How similar do you think the Brain really is?
\end{frame}




%%%%%%%%%%%%%%%%%%%%%%%%%%%%%%%%%%%%%%%%%%%%%%%%%%%%%%%%%%%%%%%%%%%%%%%%%%%%%%%%%%%%%%%%%%%%%%%%%%%



%%%%%%%%%%%%%%%%%%%%%%%%%%%%%%%%%%%%%%%%%%%%%%%%%%SOURCES%%%%%%%%%%%%%%%%%%%%%%%%%%%%%%%%%%%%%%%%%%

\section{Sources}
\begin{frame}[c,fragile,allowframebreaks]{Sources}
    \Large
    \vfill
The slides are online: \url{https://github.com/fkarg/things-to-talk-about/blob/master/htm/main.pdf} \vfill
Drop me a mail: fkarg10@gmail.com \vfill \newpage
% \bibliographystyle{plainnat}
\bibliographystyle{ieeetr}
\bibliography{references.bib}
\end{frame}


% 
\addtocounter{framenumber}{1}
\begin{frame}[standout]
    \Huge
    End
\end{frame}

\appendix
\backupbegin


\begin{frame}[c]{Master Thesis Goals}
    \large
    \begin{itemize}[<+(1)->]
        \item Using a large language model
        \item Evaluate it for data extraction tasks
        \item Fine-tuning it
        \item (maybe trying a LoRA-version for that \cite{hu_lora_2021})
        \item benchmark and compare accuracy of different model sizes and available types
    \end{itemize}
\end{frame}

\backupend




\end{document}


% TODO:

% Include Topics/Facts:
% 
% - explain SDRs
% - explain spatial & temporal pooler
% - explain HTM
% - explain reticular activating system
% - explain gradient for Autism and that it is more 'bottom up'
% - explain 'top-down' phenomena and common biases
% - explain theory: inhibited-booost: hallucination
% - The datastructure of the brain

% Roadmap (?)
%
% - First: What is Intelligence ? (Early path, why we want to know this and why this is the most likely path)
% - Introduce the biological neuron we will be talking about (multiple dendrites, one axon, ...)
% - Introduce (SDR) Sparse Distributed Representations, in-Depth
% - Introduce the CLA
% - Introduce HTMv3 !
% - Motor movements are just very strong predictions the body 'makes happen'
%   - Flow state is where everything is exactly as predicted
% - Mistakes that happen by accidential correlation: search for good example; mistaking something to be true
% - explain gradient for Einstein/Craziness, proving vs disproving in the brain -> THC is bad
% - Mistakes that happen by top-down enforcing (e.g. the dalmatine picture, scrambled words)
% - Location-based composition: This explains mind palaces!
% - Everything is a 'conceptspace' - 

% - Explain Priming based on predictive stuff (the priming-bias)
% - Explain biases based on these models
% - Explain the fact that smarter people have less Brain activity
% - Explain the bayesian-reasioning part
% - Explain how cortical columns create quite powerful models



% 
