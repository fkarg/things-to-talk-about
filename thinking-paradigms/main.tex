\documentclass[13pt,compress,ngerman,utf8,t]{beamer}
\usepackage{etex}
\usepackage[ngerman]{babel}
\usepackage{graphicx}
\usepackage[export]{adjustbox}


\usetheme[numbering=fraction, progressbar=frametitle]{metropolis}


\date{\today}
\institute{University of Freiburg}
\titlegraphic{\hspace{9cm} \includegraphics[height=2cm]{Logo-Uni-Freiburg.png}}
\graphicspath{ {./template/} }
%\titlegraphic{
%    \begin{figure}[b!]
%    \end{figure} }

\title{Denk-Paradigmen}
\author{Felix Karg}
\subject{Mathecamp}


\AtBeginSection[]
{
    \begin{frame}
        \frametitle{Inhalt}
        \tableofcontents[currentsection]
    \end{frame}
}


\begin{document}

\maketitle

\frame{\frametitle{Inhalt}\tableofcontents}


\section{Was sind das überhaupt?}


\section{Warum sollte man sich danach richten?}


\section{Welche gibt es überhaupt?}
\begin{frame}{Verschiedene Paradigmen}
    \begin{itemize}
        \item "Nichts"
        \item Karma
        \item Religion
        \item Rationalität
        \item Stoismus
        \item Maoismus
        \item Taoismus
        \item Buddhismus
        \item 'Gentleman'
        \item Utilitarismus
        \item Weitere ...
    \end{itemize}
\end{frame}


\section{Rationalität}


\section{Stoismus}


\begin{frame}[standout]
    Fragen ?
\end{frame}


\end{document}
