\section{Introduction}


\begin{frame}[c]{Key Contributions}
The key contributions of the PointNet paper are as follows:
    \begin{itemize}
        \item Design of a novel deep net architecture suitable for unordered point sets in 3D;
        \item Showing how such a net can be trained to perform 3D shape classification, shape part segmentation and scene semantic parsing tasks;
        \item Thorough empirical and theoretical analysis on stability and efficiency of the method;
        % \item Illustration of the 3D features computed by the selected neurons;
        \item Developing intuitive explanations for its performance.
    \end{itemize}
\end{frame}

\subsection{Related Work}
% \subsubsection{Point Cloud Features}
% Ignore
\subsubsection{Deep Learning on 3D Data}

\begin{frame}[c]{Deep Learning on 3D Data}
    Volumetric CNNs \cite{wu20153d, maturana2015voxnet, qi2016volumetric} apply
conventional 3d convolutional neural networks on voxelized shapes. However,
data sparsity and computation cost of 3d convolution constrain the resolution
of volumetric representation.
\end{frame}

        % Work has been done to mitigate the sparsity problem, but its still challenging for them to process very large point clouds.


\begin{frame}[c]{Multiview-CNNs}

Multiview-CNNs \cite{su2015multi, qi2016volumetric} first render 3D point cloud
in multiple 2D images and apply 2D conv nets for image classification. Given
sufficient computational resources, they achieve dominating performance due to
well-engineered image CNNs. However, it is difficult to extend
image-CNNs to other 3D or point-based tasks.

\end{frame}


\begin{frame}[c]{Feature-based DNNs}

    % \item Spectrial CNNs: spectral CNN on meshes, constrained on manifold meshes and unclear how to extend.
Feature-based DNNs \cite{fang20153d, guo20153d} extract traditional shape
features and convert 3d data to a vector before using a fully connected net for
shape classification. They appear to be limited by the representative power of
the features extracted.

\end{frame}

\subsubsection{Deep Learning on Unordered Sets}

\begin{frame}[c]{Deep Learning on Unordered Sets}
A point cloud is an unordered set of vectors from the data structure point of
view. Most works in deep learning however look at regular input structures like
ordered sequences of images, volumes or points. Unordered point clouds are
rarely considered. One recent work \cite{vinyals2015order} attempts to impose
order on unordered input sets via the attention mechanism. This work focuses on
generic sets and NLP applications, which lacks the characteristics of geometry
in the sets.

\end{frame}

% NOT: spatial transformer nets (T-Nets) related to Transformers \cite{vaswani2017attention}

\subsubsection{Based on PointNet}

\begin{frame}[c]{Based on PointNet}

There exist a number of works explaining, applying and building upon PointNet.
The influence of PointNet can furthermore be seen in the ecosystem of different
implementations and tools for visualization \cite{charlesq342022Jun,
aldipiroli2022Jun, yunxiaoshi2022Jun, Pytorch_Pointnet_Pointnet2}. Different
attempts to explain what PointNet learned \cite{zhang2019explaining,
huang2019claim} exist, and many apply PointNet to different domains and
different problems \cite{thiery2022medical, gutierrez2018deep,
triess2021survey, liang2019multi, zhang2018collaborative, mrowca2018flexible}.
\end{frame}

% as well as blog posts on Medium \cite{DataScienceUB2022Apr, Rigoulet2022Apr, Gonzales2021Dec},

\subsubsection{Derivations of PointNet}

\begin{frame}[c]{Derivations of PointNet}

PointNet, in addition to being successful itself, sees successful use as a
module similar to a convolution layer in more sophisticated neural network
architectures.
% "PointNet is becoming a module similar to a convolution layer and is used directly as the base of many netwoks" according to \cite{zhang2019explaining}.
% "Among them, PointNet is becoming a module similar to a convolution layer and is used directly as the base of many netwoks."
This can best be seen in recent architectures such as PointNet++
\cite{qi2017pointnet++}, VoxNet \cite{maturana2015voxnet}, and Syncspeccnn
\cite{yi2017syncspeccnn}. Moreover, an even larger number of architectures is
being heavily inspired by PointNet or adapts core ideas of the PointNet
architecture \cite{jiang2018sift, wang2018sgpn, yu2018pu, gutierrez2018deep,
qi2018frustum, li2018pointcnn}. This includes architectures both for 3D point
cloud data and traditionally ordered input data structures.

\end{frame}



\subsection{Problem Statement}


\section{Deep Learning on Point Sets}

\subsection{Properties of Point Sets in Rn}

\subsection{PointNet Architecture}

\subsection{Theoretical Analysis}

\begin{frame}[c]{DL on Point Sets}

\end{frame}
