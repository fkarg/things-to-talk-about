\section{Related Work}
\begin{frame}[c]{Point Cloud Features}
    \large
    \begin{figure}
        \captionsetup[figure]{labelformat=empty}
        % \captionsetup[subfigure]{labelformat=empty}
        \small
        \section{Project Results}

% \subsection{Upgrade Dependencies}
% 
% \begin{frame}[c]
%     From Django 2 to Django 3 \\
%     (Django 4 released during the project, has been decided to go to 3 in production first) \\
%     Now requires \mintinline{python}{DEFAULT_AUTO_FIELD = "django.db.models.AutoField"}
% \end{frame}

\subsection{Proper Logging}

\begin{frame}[c]{Changes to Logging}
    \large
    \begin{itemize}[<+(1)->]
        \item Logging used to be over individual `print` statements
        \item Properly defined Logging levels (debug, info, warn, error, critical)
        \item Logging to console, syslog and file
        \item Differing formatter for console and other (timestamps, ...)
        \item Rotating files: Keep last five days, overwrite after
    \end{itemize}
\end{frame}

\begin{frame}[c]{Logging Before: Example} 
    \todo{Screenshot: Logging before, Optional}
\end{frame}

\pic{Logging Now: Example}{33}

\subsection{Automatically generate Documentation}

\begin{frame}[c]{Why is this worth trying?} 
    \begin{itemize}[<+(1)->]
        \item 
    \end{itemize}
\end{frame}

\begin{frame}[c]{Automatically Generate Documentation}
    \begin{multicols}{2}
        \large
        \includegraphics[width=0.5\textwidth]{swagger} \\
        (Example) \\
        Failed, because:
        \begin{itemize}[<+(1)->]
            \item Mainly used to generate documentation for JSON endpoints
            % \item Few comments to generate documentation from
            \item 'Primitive' Views lacking important information for automatic generation
        \end{itemize}
        Was worth a try.
    \end{multicols}
\end{frame}

\subsection{What Research is the Storage used for?}

\begin{frame}[c]{What Research is the Storage used for?}
    \begin{multicols}{2}
    \includegraphics[width=0.5\textwidth]{Selection_032}
    \large
        \begin{itemize}[<+(1)->]
            \item We want to know more about our users
            \item This includes affiliation of research projects
            \item Good separation in subject areas from 'Deutsche Forschungsgesellschaft'
        \end{itemize}
    \end{multicols}
\end{frame}

\pic{Selection of DFG Subject Area upon Project Creation}{10}

\begin{frame}[c]{Selection of DFG Discipline}
    \includegraphics[width=\textwidth]{select_discipline}
\end{frame}
\begin{frame}[c]{Selection of DFG Subject without Board}
    \includegraphics[width=\textwidth]{select_subject1}
\end{frame}
\begin{frame}[c]{Selection of DFG Subject}
    \includegraphics[width=\textwidth]{select_subject2}
\end{frame}


\begin{frame}[c,fragile]{Code Requesting Fields for Boards}
    \footnotesize
    \inputminted[linenos=true]{javascript}{code/board_request.js}
\end{frame}


\begin{frame}[c,fragile]{Backend answering with available Fields}
\footnotesize
    \begin{minted}[linenos]{python}
## urls.py
path('ajax/fields/', views.view_science_fields, name="ajax_load_fields"),

## views.py
def view_science_fields(request):
    b_pk = request.GET.get('board')  # we get the pk of the selected board
    if b_pk:  # select available Fields from this Board
        fields = Science_Field.objects.filter(board__pk=b_pk) 
        return render(request, 'dropdown_list_options.html',
                      {'options': fields})
    return render(request, 'dropdown_list_options.html',
                  {'options': Science_Field.objects.none()})
\end{minted}
\end{frame}

\subsection{Diff csv to DFG schema in database}

\begin{frame}[c]{The DFG schema changes frequently}
    \large
    \begin{itemize}[<+(1)->]
        \item The DFG schema changes every four years
        \item It was last changed in 2020
        \item So it'll change again in two years
        \item Not clear how much (probably not a whole lot)
    \end{itemize}
    \pause
    So I implemented a command to compare any csv to what is currently in the
    database: \mintinline{bash}{manage.py dfg_schema_diff}
\end{frame}


% \begin{frame}[fragile]{Usage of \texttt{dfg_schema_diff}}
\begin{frame}[fragile]{Usage of \texttt{dfg\_schema\_diff}}
    \scriptsize
\begin{verbatim}
usage: manage.py dfg_schema_diff [-h] [--locale LOCALE] [--columns COLUMNS]
                     ...
                     FILE

Show difference from given file schema to DFG schema in database. By default,
ignores the now deprecated hierarchy level 1. ...

positional arguments:
  FILE                  Path to dfg_systematic.csv

options:
  -h, --help            show this help message and exit
  --locale LOCALE       Set the language of the name column to select. Can correctly select
                        both '<locale>' and 'prefLabel@<locale>' columns. (Default: 'en')
  --columns COLUMNS     Dictionary mapping columns (numbers) to expected values "level" (in
                        the hierarchy, category: 0, deprecated/ignored: 1, board: 2, field:
                        3), "notation" (e.g. 101-27), and locale translations, e.g. "en"
                        (double quotes are important!). Defaults to auto.
  ...                   ...
\end{verbatim}
\end{frame}

% \subsection{Rename Models}
% 
% \begin{frame}[c]
%     % Order -> LSDFProject
%     % PersonOrder -> ProjectRole
%     % ...
%     Worked quite well, generated some migrations
% \end{frame}

% \subsection{Attempt to rename entire App}
% 
% \begin{frame}[c]
%     Failed, ultimately unclear why, requires deep django/database knowledge.
% \end{frame}


\subsection{Attempt to properly modularize PersonForm}

\begin{frame}[c]
    Explain the problem (show exemplary code) of redundancy
    Managed to get quite far, very good (deep) learning example, and absolutely
    possible. just requires some more time than expected.
    Nags me that it's not modular yet. Some day.
\end{frame}


\subsection{Extension Requests}

\begin{frame}[c]
    WHY Extension Requests
\end{frame}

\begin{frame}[c]{Extension Requests}
    Much-requested feature.
    \todo{Overview of newly introduced routes, views, forms, models and relations}
\end{frame}

        \caption{Overview from \url{https://github.com/PointCloudLibrary/pcl/wiki/Overview-and-Comparison-of-Features}}
    \end{figure}
    Most existing point cloud features are \textbf{handcrafted for specific tasks}. \\
    \pnote{
        Die meisten zuvor existierenden Features sind \\
        nur für bestimmte Aufgaben. \\
        Skaliert sehr schlecht für neue Aufgaben.
        \par
        Die Tabelle listet ein paar davon.
    }
\end{frame}


\begin{frame}[c]{Conversion to Other Representations}
    \begin{columns}
        \column{0.1\textwidth}
        \column{0.6\textwidth}
        \includegraphics[height=0.75\textheight]{conversion}
        \column{0.3\textwidth}
        \color{ocre}
        Figures from:
        \begin{itemize}
                \color{ocre}
            \item Bunnies: CVPR presentation to \cite{qi2017pointnet} \\
            \item MVCNN: \cite{li2019angular} \\
                % \item 3D-CNN-image: \cite{roy2019ecnn} \\
            \item 3D-CNN: Supplemental to \cite{qi2017pointnet} \\
            \item Mesh-Net: \cite{feng2019meshnet} \\
        \end{itemize}
    \end{columns}
    \pnote{
        Es gibt bereits Arbeiten auf Point Cloud. \\
        \par
        Allerdings wandeln diese Punktwolken erst \\
        in andere Darstellungen um, und verwenden \\
        anschließend bereits existierende Architekturen. \\
        \\
        Und lösen jeweils nur sehr spezifische Probleme.
    }
\end{frame}


% \begin{frame}[c]{Learning on Unordered Sets}
%
%     A point cloud is an unordered set of vectors from the data structure point of
%     view. Most works in deep learning however look at regular input structures like
%     ordered sequences of images, volumes or points. Unordered point clouds are
%     rarely considered. One recent work \cite{vinyals2015order} attempts to impose
%     order on unordered input sets via the attention mechanism. This work focuses on
%     generic sets and NLP applications, which lacks the characteristics of geometry
%     in the sets.
%
%     \todo{Order Matters! Find Visualization .... none in paper}
%
%     \todo{alternatively, exclude from presentation}
% \end{frame}


\begin{frame}[c]
    \LARGE
    \begin{centering}
        Research Question: \\
        \vspace{1cm}
        Can we achieve effective \textbf{feature learning directly} on point clouds?
    \end{centering}
    \pnote{
        Frage: Können wir Feature Learning \\
        direkt auf Punktwolken erreichen? \\
        \par
        Die Antwort ist ja.
    }
\end{frame}

