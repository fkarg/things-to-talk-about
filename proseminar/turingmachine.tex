\section{Turingmaschinen}

\begin{frame}[c]{Turingmaschine: Einführung}
    \includegraphics[width=11cm]{proseminar/images/turing-machine.png}
    % Keine Formale Definition, ist bekannt oder eben nicht
    % TM hat:
    % - Unendliches Band mit Zeichen drauf
    % - Lese & Schreibkopf
    % - State-Machine
\end{frame}


\begin{frame}[c]{Turingmaschine: Beispiel}
    \includegraphics[width=10cm]{proseminar/images/tm-ex1.png} \\
    Schreibt 6 1er auf ein leeres Band.
\end{frame}


\begin{frame}[c]{Eigenschaften von Turingmaschinen}
    Relevant:
    \begin{itemize}
            \pause
        \item Äquivalent zu TM mit mehreren Spuren 
            \pause
        \item Äquivalent zu TM mit mehreren Bändern
            \pause
        \item Beliebiges Alphabet (häufig nur Binär)
            \pause
        \item Andere Berechenbarkeitsmodelle gleichmächtig
            \pause
        \item TM ist Eindeutig Definiert
            \pause
        \item Kann andere Turingmaschinen Simulieren
            \pause
        \item Halteproblem
    \end{itemize}
\end{frame}


