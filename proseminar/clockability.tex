% \section{Stempelbare Ordinalzahlen}
\section{Stempelbarkeit}


\begin{frame}[standout]
    \huge
    Können wir zu jeder Natürlichen Zahl halten?
    % Ja, einfach n Zustände.
\end{frame}


\begin{frame}[standout]
    \huge
    Können wir zu jeder Ordinalen Zahl halten?
    % Betrachten wir doch einfach mal an welchen Zahlen wir wirklich halten können.
\end{frame}



% \subsection{Stempelbare Ordinalzahlen - Einführung}
\subsection{Einführung}

\begin{frame}[c]{Stempelbare Ordinalzahlen}
    \pause
    \code{Lese das Band.
    \begin{itemize}
        \item Bei einer 1, halte.
        \item Bei einer 0, schreibe eine 1 und gehe ohne anzuhalten nach rechts.
    \end{itemize}
    }
    \pause
    $\rightarrow$ Wir halten im Schritt $\omega$. \\ \pause
    \small
    (wir haben bereits gesehen dass wir im Schritt $\omega^2$ halten können.)
\end{frame}

\begin{frame}[c]{Erkenntnisse}
    \Large
    \begin{itemize}
        \item Alle Ordinalzahlen bis $\omega^2$ sind Stempelbar.
            \pause
        \item Ist $\alpha$ Stempelbar, so auch $\alpha + \beta$; $\beta \leq \omega^2$.
            \pause
        \item Sind $\alpha$ und $\beta$ Stempelbar, so auch $\alpha + \beta$ \pause und $\alpha * \beta$.
    \end{itemize}
    \pause
    Sind das nicht bereits alle?
\end{frame}


%%%%%%%%%%%%%%%%%%%%%%%%%%%%%%%%%%%%%%%%%%%%%%%%%%%%%%%%%%%%%%%%%%%%%%%%%%%%%%%
\begin{frame}[c]{Speed-Up Lemma}
    \Large
    Wenn $\alpha + n$ Stempelbar ist, so auch $\alpha$.
    % Andersrum klar, aber eigentlich interessant aus einem anderen Grund
    % Kurzfassung: Erkennt Bandinhalt für $\alpha$, setzt gleichzeitig flag.
\end{frame}



\subsection{Lücken-Theoreme}


\begin{frame}[c]{Lücken-Existenz-Theorem}
    \Large
    Es gibt nicht-Stempelbare Lücken in den Ordinalzahlen. Um genau zu sein ist
    die erste Lücke genau $\omega$ groß. \\
    \pause

    Beweis: \pause \\
    Alle Turingmaschinen Simulieren und halten, sobald kein anderes gehalten hat.
\end{frame}


\begin{frame}[c]{Alle Turingmaschinen}
    \Large
    \alt<1,3>{Alle Turingmaschinen}{\underline{\bf Alle Turingmaschinen}} Simulieren und halten, \\
    sobald kein anderes \only<-2>{gehalten}\only<3>{\underline{\bf gehalten}} hat.
    \newline \newline \newline
    Hat es Bedeutung, davon zu sprechen?
    % Zwei dinge die erstmal komisch sind
    % Alle Turingmaschinen -> Annahme: Ja ist möglich, können alle aufschreiben und dementsprechend simulieren
    % gehalten -> starke Annahme: keine Simulierte hält. Ist berechtigt, da es weniger Ordinalzahlen gibt als mögliche TM.
\end{frame}


\begin{frame}[c]{Große-Lücken-Theorem}
    \Large
    Die Lücken werden Groß. Für jede Stempelbare Ordinalzahl gibt es mindestens
    eine genausogroße Lücke.
    \newline
    \newline
    \pause
    Beweis.
    % Zähle durch alle Ordinalzahlen, zähle $\alpha$ am rand mit solange keine
    % hält, wenn alpha durch ist: Lücke gefunden, andernfalls setze \alpha
    % zurück und fange von vorne an. Muss existieren, sonst sind wir irgendwann
    % nach den 'stempelbaren Ordinalzahlen' und können anhalten sobald wir
    % \alpha abgezählt haben.
\end{frame}


\begin{frame}[c]{Viele-Lücken-Theorem}
    \Large
    Es gibt für jede schreibbare Zahl $\alpha$ mindestens $\alpha$ viele
    mindestens $\alpha$ große Lücken in den stempelbaren Ordinalzahlen.
    % Beweis folgt.
\end{frame}


\begin{frame}[c]{Viele-Lücken-Theorem}
    \Large
    Um genau zu sein: ist $\alpha$ Stempelbar oder Schreibbar, ist die Anzahl
    der Lücken mit der Größe mindestens $\alpha$ weder Stempelbar noch
    Schreibbar.
    % Beweis:
    % Wir zählen die Lücken der Größe \alpha und clocken damit \beta.
    % Irgendwann sind wir durch, und damit also hinter den stempelbaren
    % Ordinalzahlen, wenn wir jetzt halten haben wir einen Widerspruch, also
    % kann \beta weder stempelbar noch schreibbar sein.
\end{frame}




