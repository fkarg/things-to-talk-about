\ifEnglish


\begin{frame}[c]{What is Rationality about?}
    Testframe
\end{frame}

% \begin{frame}[standout]
%     Rationality is {\emph about} winning.
% \end{frame}


\else

\section{Was ist Rationalität?}

\begin{frame}[c]{Was ist Rationalität?}
    \Large
    Grob:
    \newline
    \begin{itemize}
    \pause
    \item Die Welt verstehen    \only<3->{(Epistemic Rationality)}
    \newline
    \pause
    \pause
    \item Ziele erreichen       \only<5->{(Instrumental Rationality)}
    \newline
    \end{itemize}
    \pause
    \pause
    Kurz: Rationalität entwickelt unsere Entscheidungen bezüglich Denken und Handeln weiter.
\end{frame}


\begin{frame}[c]{Was ist Rationalität?}
    \Large
    Nutzen von Erkenntnissen aus anderen Bereichen, z.B.:
    \begin{itemize}
    \pause
    \item Statistik         \only<3->{(Bayes' Theorem)}
    \pause
    \pause
    \item Philosophie       \only<5->{(Effektiver Altruismus)}
    \pause
    \pause
    \item Entscheidungstheorie  \only<7->{(Occams Razor)}
    \pause
    \pause
    \item Kognitionswisschenschaften    \only<9->{(System I/II)}
    \pause
    \pause
    \item Kritisches Denken \only<11->{(Wissenschaftliche Methode)}
    \end{itemize}
\end{frame}

% Wertetheorie, Erkenntnistheorie, Metaphysik

\begin{frame}[c]{Was tun mit den Erkenntnissen?}
    \Large
    \pause
    Optimieren von ...
    \begin{itemize}
    \pause
    \item Strukturen
    \pause
    \item Denkprozessen
    \pause
    \item Verfahrensprozessen
    \pause
    \item ...
    \end{itemize}
\end{frame}





\fi
